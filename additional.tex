\chapter{Additional for Dissertation}
\emph{The last chapter containing some additional things needed for a complete 
dissertation document required for a promotion at the University of Groningen.}
\null
\vfill

\newpage
\section*{Dutch Summary}
\addcontentsline{toc}{section}{Dutch Summary}
Systems genetics is the inter disciplinairy field which deals with the consequences of genetic 
variation on a biological system. The goal of systems genetics is to partition this variation 
into three major categories: Genetic, Environment and Error variation, and explain how complex
phenotypes arise from a combination of these three major factors.  Natural variation (or 
genetic pertubation) is used to interogate the genetic component of phenotype variation. 
Combined with environmental pertubation we can investigate the influence different 
environments and the interaction between genetics and environment. Experimental design and 
statistics are used to minimize and estimate error variance.

Pheno2Geno (Chapter 2) deals with the creation of genetic maps from large scale omics data. 
The theory of genetic map construction is ~100 years old. Software available for genetic map 
construction has not been adapted yet to make use of new technologies such as multi threading 
or cluster computing. Pheno2Geno aims to provide analysis of data from tilling arrays or RNAseq 
to generate gene based expression markers and create high density genetic maps.

Chapter 3 describes the implementation of the Multiple QTL mapping routine by R. C. Jansen into 
R/qtl. Adding a 'new' algorithm to the R/qtl toolset which aims to provide a range of QTL 
mapping tools for inbred crosses. R/qtl is the basis of a toolset build around a unified 
data structure allowing easy adaptation and extension of the software. R/qtl allows researchers 
to analyse data from different sources or to quickly compare different approaches.

In chapter 4 we show case our ideas for a generic storage and computation platform for systems 
genetics. Our demo system xQTL workbench is currently being used as a backend to the 
WormQTL and WormQTL-HD database. xQTL workbench allows users to store and share their data in 
a local or web environment, and run analysis across datasets using the power of distributed 
computing. It comes standard with QTL mapping tools such as: R/qtl, PLINK and qtlbim but also
provides a web inferfaces, data importers, APIs and visualizations.

Current work on using differences in correlation to generate interaction networks and detect 
cell type specific QTL effects is described in chapter 5. Correlated Traits Locus analysis 
(or CTL mapping) enables researchers to find genetic loci controlling correlation differences 
in segregating phenotypes. A variation on this method has proven valuable in discovering cell 
type specific eQTL effects. These effects can be used to untangle cell mixtures seen in whole 
blood gene expression data.

%In short this proposed method creates a proxy for cell counts in an unrelated cohort. this proxy 
%is then used during meta analysis as artifical cell count across many different cohorts. The 
%interaction effects between QTL and Proxy are used to assign a cell type to each eQTL. In the final 
%stage diseases in which cell type enrichment is prominent are tested by using publicly available data.

Last we have some additional required chapters, such as this summary in Dutch, a CV and publication 
list and bibliography.\\

I hope you enjoy reading this thesis as much as I enjoyed creating it during the last four years,\\\\

Danny Arends (Aug 2013)

\newpage
\section*{Abbreviations and Acronyms}
\addcontentsline{toc}{section}{Abbreviations and Acronyms}
{\footnotesize
\begin{tabular}{ l l }
API:         & Application Programming Interface\\
BC:          & Backcross \\
bp:          & Base pair(s) \\
cM:          & centi Morgan \\
CPNN:        & Collaborative computing project for NMR\\
CSV:         & Comma separated values\\
CTL:         & Correlated traits locus \\
DesignGG:    & Experimental design of genetical genomics software\\
DRY:         & Principle of don't repeat yourself\\
DSL:         & Domain Specific Language\\
EBI:         & European Bioinformatics Institute\\
FINDIS:      & Finish disease database\\
GEN2PHEN:    & EU project to unify human and model organism genetic variation databases\\
GMOD:        & Generic model organism database project\\
GUI:         & Graphical user interface\\
GWAS:        & Genome Wide Association Study\\
GWL:         & Genome Wide Linkage analysis\\
HGVBaseG2P:  & Human genome variation database of genotype-to-phenotype information\\
HTML:        & Hypertext markup language\\
IDE:         & Integrated Development Environment\\
JAR:         & Java Software Archive\\
LGPL:        & Lesser general public license\\
MAGE-TAB:    & Microarray gene expression tab delimited file format\\
Mbp:         & Mega base pairs = 1.000.000 bp \\
MOLGENIS:    & Molecular genetics information systems toolkit\\
NordicDB:    & Nordic Control Cohort Database\\
OBF:         & Open Bioinformatics Foundation\\
OntoCAT:     & Ontology common API toolkit\\
PEAA:        & Patterns for enterprise application architecture\\
QTL:         & Quantitative trait locus\\
RDF:         & Resource description format\\
REST:        & Representative state transfer web services\\
RIL:         & Recombinant inbred line \\
SNP:         & Single Nucleotide Polymorphism\\
SOAP:        & Simple Object Access Protocol\\
SQL:         & Structured Query Language\\
UML:         & Uniform data Modeling Language\\
WAR:         & Web Application aRchive file\\
XML:         & Extensible Markup Language\\
XGAP:        & Extensible genotype and phenotype software platform. 
\end{tabular}
}
\newpage

\section*{Acknowledgements}
\addcontentsline{toc}{section}{Acknowledgements}
Knowledge is in the end based on acknowledgement\\
- Ludwig Wittgenstein (1889 - 1951)\\\\

There are many people who need to be thanked and acknowledged, 
here are the people who made the cut:

Ritsert C. Jansen to whom I owe my life in science, thanks you for giving me this 
chance to work in your group and develop my skills. Anna Mulder my girlfriend for 
all the times I left you and Oscar (our cat) for my science endavours. Thank you 
for your trust in me and giving me a home to come home to. 
Pjotr Prins ..
The GBIC group for 
putting up with me. Yang Li for the vibrant discussions, Joeri v/d Velde and 
Konrad Zych in particular. The Guys from the GCC department for all the fun times 
we had.Lude Franke and Harm-Jan Westra for the nice collaborations and help with 
the last chapter. Wilko ligterink, Henk Hilhort and Ronny Joosen for doing the 
Arabidopsis experiments and letting me use their data. The University of Groningen 
for educating me, and last the University Medical Centre Groningen for the pleasant 
atmosphere and good lunch.

\newpage

\section*{Curriculum Vitae}
\addcontentsline{toc}{section}{Curriculum Vitae}
Danny Arends was born at 15th of Juli 1983 in the city of Zwolle located in the heart of the 
Netherlands.\\

Highschool\\\\

University\\\\

PHD\\\\

\newpage

\section*{List of Publications}
\addcontentsline{toc}{section}{List of Publications}
\subsection*{Authored:}
  \authors{Danny Arends*, Pjotr Prins*, Ritsert C. Jansen and Karl W. Broman}\\
  R/qtl: high throughput Multiple QTL mapping\\
  \bold{Bioinformatics} (2010)\\\\
  \authors{Ronny V. L. Joosen*, Danny Arends*, Leo Willems, Wilco Ligterink, Henk Hilhorst, Ritsert C. Jansen}\\
  Visualizing the genetic landscape of Arabidopsis seed performance\\
  \bold{Plant Physiology} (2011)\\\\
  \authors{Danny Arends*, K. Joeri van der Velde*, Pjotr Prins, Karl W. Broman, Steffen Moller, et al.}\\
  xQTL workbench: a scalable web environment for multi-level QTL analysis\\
  \bold{Bioinformatics} (2012)\\\\
  \authors{L. Basten Snoek*, Joeri Van der Velde*, Danny Arends*, Yang Li*, Antje Beyer, Mark Elvin, 
           Jasmin Fisher, Alex Hajnal, Michael O Hengartner, et al.}\\
  WormQTL: Public archive and analysis web portal for natural variation data in Caenorhabditis spp\\
  \bold{Nucleic Acids Research} (2012)\\\\
  \authors{Ronny V. L. Joosen*, Danny Arends*, Yang Li*, Leo Willems, Joost J.B. Keurentjes, Wilco Ligterink, 
           Ritsert C. Jansen, Henk Hilhorst}\\
  Identifying genotype-by-environment interactions in the metabolism of germinating Arabidopsis seeds 
  using Generalized Genetical Genomics\\
  \bold{Plant Physiology} (2013)

\subsection*{Co-Authored:}
  \authors{Morris A Swertz, Martijn Dijkstra, Tomasz Adamusiak, Danny Arends, et al.}\\
  The MOLGENIS toolkit: rapid prototyping of biosoftware at the push of a button\\
  \bold{BMC Bioinformatics} (2010)\\\\
  \authors{Morris A Swertz, K Joeri van der Velde, Bruno M Tesson, Danny Arends, et al.}\\
  XGAP: a uniform and extensible data model and software platform for genotype and phenotype experiments\\
  \bold{Genome Biology} (2010)\\\\
  \authors{Klaus Schughart, Danny Arends, P. Andreux, R. Balling, Pjotr Prins, et al.}\\
  SYSGENET: a meeting report from a new European network for systems genetics\\
  \bold{Mammalian Genome} (2010)\\\\
  \authors{Rudolf SN Fehrmann, Ritsert C. Jansen, Jan H. Veldink, Harm-Jan Westra, Danny Arends, et al.}\\
  Trans-eQTLs Reveal that Independent Genetic Variants Associated With a Complex Phenotype Converge on 
  Intermediate Genes, with a Major Role for the HLA\\
  \bold{Plos Genetics} (2011)\\\\
  \authors{Caroline Durrant, Morris A. Swertz, Rudi Alberts, Danny Arends, Klaus Schughart, et al.}\\
  Bioinformatics tools and database resources for systems genetics analysis in mice - a short review 
  and an evaluation of future needs\\
  \bold{Briefings in Bioinformatics} (2011)\\\\
  \authors{K. Joeri van der Velde*, Mark de Haan, Konrad Zych, Danny Arends, L. Basten Snoek, 
           Jan E. Kammenga, Ritsert C. Jansen, Morris A. Swertz and Yang Li}\\
  WormQTLHD - a web database for linking human disease to natural variation data in C. elegans\\
  \bold{Nucleic Acids Research} (2013)

\subsection*{Under Review / In Press:}
  \authors{Konrad Zych, K. Joeri van der Velde, Ronny V. L. Joosen, Wilco Ligterink, Ritsert C Jansen 
           and Danny Arends}\\
  Pheno2Geno - High throughput generation of genetic markers and maps from molecular phenotypes\\
  \bold{BMC Bioinformatics} (201X)\\\\
  \authors{Steffen M\"oller, Ren\'e Sch\"onfelder, Hajo Krabbenh\"oft, Benedikt Bauer, Yask Gupta, 
           Pjotr Prins, ..., Danny Arends, et al.}\\
  TiQS: web environment for expression QTL analysis\\
  \bold{BMC Bioinformatics} (201X)\\\\
  \authors{Elisabeth M. Lodder, Brendon P. Scicluna, L. Beekman, Danny Arends, et al.}\\
  Multiple QTL mapping of cardiac collagen deposition in an F2 population of Scn5a mutant mice reveals 
  interaction between Fgf1 and Pdlim3, Gpr158 \& Itga6\\
  \bold{Genome Research} (201X)

\subsection*{In Preparation:}
  \authors{Danny Arends, Pjotr Prins, Harm-Jan Westra, Yang Li, Lude Franke and Ritsert C. Jansen}\\
  Correlated Traits Locus mapping\\
  \bold{Unknown} (201X)

\subsection*{Acknowledged in:}
  \authors{Yang Li, Morris A Swertz, Gonzalo Vera, Jingyuan Fu, Rainer Breitling and Ritsert C Jansen}\\
  DesignGG: an R-package and web tool for the optimal design of genetical genomics experiments\\
  \bold{BMC Bioinformatics}, 10:188 (2009)\\\\
  \authors{Yang Li, Rainer Breitling and Ritsert C. Jansen}\\
  Generalizing genetical genomics: getting added value from environmental perturbation\\
  \bold{Trends in Genetics}, 24:518-524 (2008)


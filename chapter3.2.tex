\section{Metabolites in a DesignGG experiment}
A complex phenotype such as seed germination is the resultant of several genetic and environmental 
cues and requires the concerted action of many genes. The use of well-structured recombinant inbred 
lines in combination with omics analysis can help to disentangle the genetic basis of such 
quantitative traits. This so called genetical genomics approach can effectively capture both 
genetic (G) and epistatic interactions (G:G). However, to understand how the environment interacts 
with genomic encoded information (G:E) a better understanding of the perception and processing of 
environmental signals is needed. In a classical genetical genomics setup this requires replication 
of the whole experiment in different environmental conditions. A novel generalized setup overcomes 
this limitation and includes environmental perturbation within a single experimental design. 

We developed a dedicated QTL mapping procedure to implement this approach and used existing 
phenotypical data to demonstrate its power. Additionally, we studied the genetic regulation of 
primary metabolism in dry and imbibed Arabidopsis seeds. Many changes were observed in the 
metabolome which are both under environmental and genetic control and their interactions. 
This concept offers unique reduction of experimental load with minimal compromise of statistical 
power and is of great potential in the field of systems genetics which requires a broad 
understanding of both plasticity and dynamic regulation.

\subsection{Background}
The use of natural variation to disentangle the genetic (G) mechanisms underlying phenotypic differences 
has been very successful both in crop plants and in the model plant Arabidopsis (Arabidopsis thaliana; 
\cite{Alonso-Blanco:2009}. Most of the variation within wild or domesticated plant species is of 
quantitative nature determined by G polymorphisms at multiple loci. Such quantitative trait loci (QTL) 
can beanalyzed efficiently using experimental mapping populations such as recombinant inbred lines (RILs)
derived from directed crosses. Nowadays, many wellstructured RIL populations are available, often 
accompanied with detailed studies of phenotypic variation \cite{Mitchell-Olds:2006}. The complexity 
of quantitative traits is further determined by the interactions between genomic loci (i.e. epistasis) and 
between the genotype and the environment (genetic X environmental [G:E]). While epistasis can be effectively
identified in QTL analyses, albeit with lower power than main effects, the detection of G:E interactions 
requires experimentation in multiple conditions of interest. Because of the large population sizes often 
needed to obtain sufficient statistical power for QTL detection, G:E interactions are usually ignored in 
experimental setups. However, a better understanding of the perception and processing of environmental (E)
signals is greatly needed, because interactions provide important insights in adaptation mechanisms and
evolutionary constraints such as balancing and disruptive selection. To obtain a more detailed view of the
molecular mechanisms underlying phenotypic variation, genetical genomics studies, in which molecular traits
are genetically analyzed, have been successfully applied to enhance a directed strategy to identify causal
relationships \cite{West:2007, Keurentjes:2007, Kliebenstein:2006, Rowe:2008}. The observed phenotype is often the resultant 
of a functional cascade of gene transcription followed by protein translation and modification, which 
finally leads to a highly dynamic metabolome underlying emergent properties \cite{Kooke:2012}. 
With the technological advances made in genomic analytical platforms, such as transcriptomics, proteomics, 
and metabolomics, the large-scale, high-throughput analyses needed for quantitative G approaches have 
become feasible \cite{Jansen:2001a}. 

Incorporating developmental and E  perturbation in the often expensive and laborious omic analyses, an 
alternative experimental setup, coined generalized genetical genomics (GGG), using balanced fractions 
of a RIL population has been proposed \cite{Li:2009}. It provides a cost-effective experimental setup 
for hypothesis-generating research in multiple environments. Such an approach aims for the creation of 
subpopulations of RILs, one for each environment to be tested, with an optimal distribution of parental 
alleles over all available markers \cite{Li:2009}. When these subpopulations are subjected to E 
perturbation, the emerging phenotypes can be explained by several sources of variation: G variation, 
E variation, and G:E variation. Whenever the resulting phenotype is not or only mildly affected by E 
interactions (G:E), the analysis of the different subpopulations can be combined, gaining the full power 
of a complete population. However, when a trait shows strong G:E interaction (e.g. those that only express 
G variation in specific environments), the power to detect QTL is dependent on those subpopulations 
expressing the G variation. Although G:E interactions have been detected previously in genetical 
genomics studies for expression \cite{Li:2006, Smith:2008, Gerrits:2009} and metabolite content 
\cite{Zhu:2012} by analyzing all lines in a population under different environments, the GGG concept 
offers an effective way of studying a combination of G and E perturbations and is of great potential 
in the field of systems genetics, in which a broad understanding of both plasticity and dynamics is 
required \cite{Li:2008}. The fundamental basis of the experimental design and data analysis using a 
full model ($Y = E + G + G:E + e$), where $Y$ is the observed phenotype and $e$ is residual error, 
is generally valid and frequently used \cite{Churchill:2002, Li:2006,Gerrits:2009}. As a proof of 
principle, we present experimental data on the G regulation of primary metabolism in dry and imbibed 
Arabidopsis seeds using a GGG design and discuss the application and implications of such a strategy.

Plants are extremely rich in biochemical compounds, and major roles in plant development, adaptation, 
and defense have been identified for biosynthesis pathways and their products (Binder, 2010). The 
biosynthetic pathways of primary metabolites are well studied and often well conserved between 
different taxa \cite{Peregrin-Alvarez:2009}. Nonetheless, quantitative variation for many of 
these compounds can be observed between natural variants, which might be reflected in their different 
growth characteristics. The analysis of single-gene mutants, for example, has unraveled many key 
components in biochemical pathways and has demonstrated their role in phenotypic traits 
(Fiehn et al., 2000). In Arabidopsis, G variation for many of its metabolic compounds has been 
observed \cite{Kliebenstein:2001, Rowe:2008, Keurentjes:2006}, but G:E interactions 
were ignored in these studies and only addressed by Chan et al.\cite{Chan:2011}. Metabolic profiling at different 
growth stages has further revealed important fluxes that regulate plant development and adaptation 
\cite{Oliveira:2010}. Using the accumulated historical mutations that occur in natural 
variants in combination with metabolic profiling in a generalized design offers the unique possibility 
of identifying G effects over a series of developmental stages. Here, we report on the interaction of 
four different physiological environments (i.e. developmental stages) in dry and imbibed seeds with 
two founder genotypes in a RIL population. To detect the majority of the most prominent primary 
metabolites, we used gas chromatography-mass spectrometry of polar extracts \cite{Roessner:2000, 
Lisec:2008}. These include essential metabolites such as sugars, amino acids, and organic 
acids, which are key compounds in reserve storage and catabolism, growth, and energy metabolism.

The switch from a dry seed, which is equipped for optimal survival and storage of reserves, toward an
imbibed seed, in which energy needed for germination is released and which prepares for autotrophic pro-
duction, is remarkable. Reserves that have been stored during seed maturation are degraded and remobilized
during germination \cite{Bewley:1997,Shu:2008}, a process that is heavily influenced by the capacity 
of carbon/nitrogen partitioning of a maturing seed \cite{Dowdle:2007}. Arabidopsis mutants affected 
in their oil reserve content or its mobilization show delayed but not full inhibition of germination 
(Kinnersley and Turano, 2000; Bouché and Fromm, 2004; Shu et al., 2008; Kelly et al., 2011). This suggests 
an additional metabolic switch that occurs during seed desiccation after seed maturation involving a change 
from accumulation of oil and storage proteins to the synthesis of free amino acids, sugars, fatty acids, 
and their degradation products functioning to prepare for rapid metabolic recovery during imbibition 
\cite{Fait:2006, Angelovici:2010}. Imbibition of mature seeds specifically shows reduction of 
the metabolites that accumulate during the desiccation period. Upon germination, an increase of many 
metabolites, including amino acids, sugars, and organic acids, can be observed again, which reflects 
the increase of autotrophic activity\cite{Fait:2006}. Profiling the primary metabolome over different 
developmental stages in a mapping population is therefore expected to reveal the dynamics of G regulation 
of many of these important processes. We will demonstrate here that much of the observed variation in 
biochemical profiles can be attributed to genotype-by-environment interactions, which can be effectively 
identified in a GGG approach.

\subsection{Results}
In the experimental setup of this study, the E variation is defined as variation observed between the four
developmental stages (PD, AR, 6H, and RP). Significance thresholds, determined by permutation analysis
($n = 1,000, P < 0.01$) for each metabolite, ranged from LOD 3.43 to LOD 3.50 and was stringently set to 
LOD 4 for all analyses. Mapping resulted in 120 significant QTLs in the G component for 83 metabolites and 
31 G:E QTLs for 27 metabolites, ranging from one to four QTLs per metabolite. Thirteen of the G:E QTLs are
significant in the G component as well. For 66 metabolites, no significant QTL was detected. Clustered 
heat maps for both the G and the G:E QTL profiles were created (Supplemental Figs. S4 and S5).

To test the performance of the generalized mapping procedure, QTLs detected in individual environments
using the linear model $Y = G + e$ were compared with QTLs detected in the combined mapping 
approach (using the linear model $Y = E + G + G:E + e$; Fig. 2; Supplemental Table S1).
QTLs were binned in upper or lower chromosome arms to reduce the effects of small positional shifts.
Results were plotted in a network, with nodes representing QTLs connected with edges to nodes repre-
senting the mapping populations in which they were detected (Fig. 2). QTLs are grouped in three sections
according to their detection in the different mapping procedures. The middle section shows 73 QTLs that
were detected in both the $Y = E + G + G:E + e$ model and in one or more single-environment mappings 
using the $Y = G + e$ model. This shows that most of the G variation present in the single environments 
can effectively be captured by using the generalized model.

The presence of 60 QTLs that were only significantly detected in the $Y =E + G + G:E + e$ model 
(right section) shows the combined power of the generalized approach and the usage of more genotypes. 
These QTLs are not detected in the single-environment mapping in which only 41 individuals were used.
Combining all data across all environments in the linear model increases power to detect QTLs, but it
should be noted that there are also 20 minor QTLs (leftsection) that are only significant in the single
environment mapping using the $Y = G + e$ model. These QTLs are not detected in the $ Y = E + G + G:E + e$
model. This can be explained by two factors: (1) environments in which the G variation is not expressed
introduce noise in the experimental data and thereby decrease mapping power, and (2) deviations from a
balanced allele distribution in the different subpopulations can introduce some stochasticity around the
threshold level, although this is not the case in our data.

Importantly, all major-to-moderate-effect-size QTLs could be detected using the generalized model, even
when these QTLs were not detected in the separate environment models. Although it is difficult to 
compare power with the latter models, because population sizes differ, the generalized design 
efficiently identifies all relevant QTLs, which were detected by the four separate models, and in addition, 
it detects G:E interactions. In a general exploratory study, the reduction in experimental burden therefore 
amply outweighs the incidental failure to detect the limited number of small-effect QTLs. The application 
of a GGG design can thus be an important advancement in evolutionary and ecological studies assessing 
the contribution of G and E effects to natural variation in life history traits.

For breeding purposes, the allelic effect size is an important measure, and differentiation of the 
environment in which the allelic effect is expressed can be very useful. In the generalized setup, 
the allelic effect size of those metabolites with significant QTLs is separated per environment 
(Supplemental Files S4 and S5). For every QTL that is consistently detected in all four conditions, 
a LOD score for G effect (Fig. 3,x axis) is obtained from full-model mapping. For these QTLs,
normalized allelic effect sizes are calculated by Zscore transformations for each environment 
(Fig. 3, y axis). QTLs detected in the G component of the linear model (Fig. 3A) show an expected 
linear relationship between LOD score and effect size in all measured environments. This correlation 
is much weaker for QTLs detected in the G:E component of the linear model (Fig. 3B) because the G 
variation is not expressed in all environments. QTLs of metabolites with strong G:E interaction, 
therefore, display larger effect sizes in fewer environments compared with G-component QTLs of 
similar significancelevels.

Clearly, the choice of environments used in these studies is crucial\cite{Li:2008}. Limited power 
can be expected when environments vary too much and no overlapping G variation is present, and 
contrarily, there is hardly any additive value of the design when using very similar environments. 
In this study, we carefully selected four biologically relevant developmental stages of seed germination 
with expected variation in metabolite levels to different extent and consider them as an E factor in 
the follow-up statistical analysis. The selected developmental stages start from PD dry seeds to 
seeds at the point of RP. The first two stages, being freshly harvested PD and AR nondormant dry 
seeds, respectively, are expected to comprise a very similar metabolome, as most, if not all, metabolic
fluxes are arrested in the dry seed. The other two stages represent 6H seeds and seeds at RP, respectively.
Different levels of E variation were obtained and could be mapped by the G and/or G:E component of 
the linear model.

\subsubsection{Confirmation of Metabolic QTLs}
To independently confirm the effect of a single locus, it must be isolated and tested in an isogenic 
background. Several methods can be followed to perform such an independent confirmation of QTLs. A 
powerful approach is the use of residual heterozygosity in early generations of RILs. The Bay-0XSha 
RIL population(420 lines in total) was genotyped at F6, in which approximately 97\% homozygosity is 
reached in each line. This resulted in the presence of residual heterozygosity in at least a single 
RIL at almost all genome positions. Those heterozygous regions are segregating in a Mendelian fashion 
in the next generation and can be used to confirm QTL positions, as it provides a possibility to
study both parental alleles at the locus of interest in an otherwise homozygous background 
\cite{Tuinstra:1997}. In a heterogeneous inbred family (HIF), those heterozygous regions are fixed, 
and two separate lines containing the alleles of both parents, respectively, are maintained.

HIF312 and HIF214 are segregating for regions at the top of chromosomes 4 and 5 (Fig. 6A), respectively,
and cover the region in which the two major metabolite hotspots were detected. AR dry seeds were used to
profile the HIFs for metabolic content because many of the QTLs detected in this region showed a 
large-effect size at the dry seed stages. Significant differences between parental alleles using four 
replicates were defined by a two-tailed Student's t test($P < 0.05$). In total, 34 out of 64 QTLs could 
be confirmed using this approach (Supplemental Fig. S8). For maltose, for instance, two QTLs with 
opposite direction were found (Fig. 6B), which could both be confirmed using the two distinct HIFs 
(Fig. 6C). In a number of cases, a HIF effect was observed that was not detected significantly
in the RIL population (e.g. digalactosylglycerol). This might be the result from the higher power in 
near isogenic lines due to the absence of epistatic interactions \cite{Keurentjes:2007a}. Nonetheless, 
a substantial number of QTLs could not be confirmed by the HIF lines. The enrichment for small-effect 
QTLs in the unconfirmed class suggests that four replicates generate insufficient power to identify 
significant differences for these metabolites in the HIF experiments, although we cannot rule out that 
they are false positives from the QTL analysis. Furthermore, QTLs depending on epistatic interactions 
cannot be detected in some near-isogenic lines. In addition, a number of QTL support intervals are 
broader than the region covered by the HIF, and thus, the causal G polymorphism within the QTL interval, 
but outside the region covered by the HIF, would have been missed.

The analyses of the HIF lines indicate that most of the large-effect QTLs can be accurately detected 
using a generalized genomics approach. Although an underestimation of small-effect QTLs can be expected, 
this is largely compensated by the higher power of detecting G and E interactions


\chapter{CTL and G:E QTL mapping}
\label{chap:ctlmapping}

\emph{In this chapter we develop a new methodology to be used in quantitative genetics 
called Correlated Traits Locus (CTL) mapping, a method complementairy to QTL mapping. 
Where QTL associates differences in mean, CTL, associate differences in correlation to 
genetic variation, i.e. CTL identify regions in the genome for which one genotype leads 
to correlated expression between a pair of traits, while the other genotype shows none 
(or significantly different) correlation.}

\null
\vfill

\begin{myexampleblock}{In press:}
  \authors{Danny Arends, Pjotr Prins, Yang Li, Lude Franke and Ritsert C. Jansen}\\
  \emph{CTL mapping}\\
  \bold{Unknown} (XXXX)
  \authors{HarmJan Westra*, Danny Arends*, ... ,  Ritsert C. Jansen and Lude Franke}\\
  \emph{Cell type specific eQTL mapping}\\
  \bold{Nature Methods} (2013)
\end{myexampleblock}

\newpage

\section{What is a CTL?}
\lipsum[1]

\section{Combining CTL and QTL information}
\lipsum

\section{Cell type specific eQTL mapping in human GAWA data}
\label{sec:cellspecificeqtl}


\chapter{Introduction}
\emph{The introduction of this thesis, background information and the outline of the next chapters.}

\null
\vfill
\newpage

\section{Introduction}
Systems genetics is the inter disciplinairy field which deals with the consequences of genetic 
and natural variation on a biological system. Phenotype diversity is created by a mix of variation 
at the DNA level (Nature), the environment (Nurture), and an error component. Partitioning of this 
variance into these three sources is of critical importance to understanding the contribution of 
Nature and Nurture to the biological processes we observe. Understanding how a system works, allows 
us to change the system to our needs. We can use our knowledge about DNA variation to: optimize 
breeding of animals/plants, exploring underlying causes for human diseases, improved production 
of chemicals, and so forth.

The use of computational tools in understanding how Nature and Nurture intertwine is the 
focus of this systems genetics. Variation observed at the DNA level can be measured using 
genotyping (AFLP, RFLP, SNP chips, nextGEN sequencing) techniques, allowing us to create 
detailed maps of DNA inheritance within a population and track the origin of DNA. 

On the other side at the phenotype end of the spectrum new high throughput technologies 
have been developed to mass produce phenotypes measurements using automated phenotyping. 
Automated phenotyping allow measurements at all the different endophenotype levels such 
as: metabolome, proteome, transcriptome, proteomics. New technologies such as microarrays 
and RNAseq now enable us to measure all levels from the DNA to the observed phenotype at 
a fraction of the costs.

How the activity of DNA is transfered and modified by signals from the enviroment is of 
critical importance to our understanding of systems genetics. Variation observed at levels 
'further away' from the genome such as: transcriptome, proteome, metabolome, all the way 
up to classical phenotypes is a complex interplay between DNA and environment. This 
interplay is challenging to unravel because of multiple factors:\\
\begin{itemize}
\item Complex phenotypes are caused by multiple genetic loci, interactions between these, 
modulated by the environment.
\item The more factors underlying a complex phenotype, the larger the samplesize required 
to determine which genetic factors are involved.
\item Data collection however remains limited by resources available (money, sample size, labour, etc)
\item Large scale data collection puts more and more demands on our current infrastructure for 
storage and sharing of this data.
\item Computational issues arise in the analysis of such big data sets (CPU time, RAM requirements, 
Power consumption).
\item Biologie.
\end{itemize}

The following chapters of this thesis will go into our solutions to these issues, infrastructure 
created, methodologies developed, and show their application in current research.

\subsection{Background}

Genetics is the science of heritability. The field of genetics is a relatively new field, 
and starts when Gregor Mendel publishes his work on the inheritance of traits in pea plants. 
His work described in 'Versuche \"uber Pflanzen-Hybriden' in 1865-1866 \cite{Mendel:1866} 
gives us Mendels Laws of Inheritance. He observes that crossing two plants with different 
colors of flowers will lead to an offspring population with predictable flower color 
ratio's. Additionally he's fortunate enough to have traits which are inherited independently, 
and thus conclused that a unit of inheritance must exist, and is passed from parent to 
child in an independant fashion. However Mendels work was not recognized untill 30 years later, 
when his work is 'rediscovered' by Hugo de Vries, Carl Correns en Erich von Tschermak who 
define the rules for Mendielian genetics \cite{deVries:1889} providing biologist for the first 
time with a mathematical framework to study heritability of phenotypes caused by a single gene.

50 years after the Mendels laws, genetics is to change again by the carefull observations of 
Thomas Hunt Morgan. He observes that some traits in his Drosophila flies are inherited together, 
in conflict with Mendels laws. he calls this phenomenon genetic linkage. Sturtevant a student 
of Morgan uses this observed linkage between phenotypes as a distance measurement between 
units of inheritance. Leading to the first genetic map of Drosophila.

However at this point in time, the molecular mechanisms were still not known. Genetics was based 
on observing segregating phenotypes, and then placing these into a known map relative to other 
phenotypes. 


The dscovery of DNA as the carrier of heritability and the discovery of its structure by Watson 
and Crick.in the mid 1900 was the long sought after molecular basis for genetics. After the 
discovery of the first restriction enzymes it also became possible to cut DNA and look at the 
different (lengths of) fragments produced.

In the end of the 1980s everything was set for the next step in genetics, and 
a series of three papers published about the use of RFLP linkage maps to map quantitative 
traits by two (now famous) authors: David Botstein and Eric Lander 
\cite{Lander:1986, Lander:1987, Lander:1989}. For the first time in history heritable traits 
could be mapped to genome locations using DNA restriction enzymes and PCR. Together whith knowledge 
of meiotic randomization this provides the experimental background for statistical analysis 
of inbred populations. Using association mapping is still the foundation of genetics research 
today, although more sophisticated tools and algorithms are used.

Multiple QTL mapping - Jansen et al. (1993)\\
Composite interval mapping -  Zeng et al. (1993)\\
R/qtl (2003)\\
Human genome project 2000-2003\\
First human Genome Wide Association Study (2005)\\
Encyclopedia Of DNA Elements (Encode) (2003, 2007-Current)


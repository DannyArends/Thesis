\chapter{Introduction}
Phenotype diversity is created by variation at the DNA level, The most prominent source of variation comes 
from the mixing of DNA by and from our parents. Understanding how this variation observed at the DNA level 
is transfered and modified by signals from the enviroment is of critical importance to our understand of 
biology and the role of genetics. All variation observed at higher levels such as: transcriptome, proteome, 
metabolome, celltype, organ all the way up to phenotypes exhibited by the whole organism stem from the 
variation in DNA and Environment. This mixture of variation in environment DNA proves to be challenging 
because of multiple factors:

1) Many factors are involved when looking at complex phenotypes
2) Data collection is limited by resources available
3) The large scale storage and sharing of data
3) Computational issues when analysis big data
4) Methodological issues

This thesis will highlight some of the advancements made in these fields, but first we need to set the stage.

Before Thesis

The field of genetics starts with the monk Gregor Mendel (1870)
Genetic maps are from Tomas Morgan hunt (1913)
Discovery of DNA
Using RLPF markers to build genetic maps - Botstein and Lander (1983)
'normal' analogous to ANOVA - Botstein and Lander (1989)
Multiple QTL mapping - Jansen et al. (1993)
Composite interval mapping -  Zeng et al. (1993)
non parametric modified Kruskal-Wallace - Kruglyak and Lander (1995)
'binary' analogous to logistic regression - Xu and Atchley (1996)
'two-part' analogous to survival - Boyartchuk (2002)


This thesis

The first chapter: Pheno2Geno deals with the creation of genetic maps from large scale omics data. The theory 
of genetic map construction is ~100 years old and was invented in a time where 100 phenotype markers were a 
dense genetic map. In recent years software has been developed to do these kinds of analysis. However looking 
at the genetic maps available we still see that many still rely on 80s and 90s technology. Pheno2Geno aims to 
provide a platform to cope with new kinds of big data available now-a-days such as tilling arrays, RNAseq and 
next generation sequencing to generate high density genetic maps from expression markers. It is writen in the 
R language for statistical computing.

The second chapter is the continuation of the Multiple QTL mapping work done by R. Jansen, we incorporated his 
method in the R/qtl package.Adding a 'new' algorithm to the QTL toolbox of R/qtl which aims to provide a range 
of QTL mapping tools basid on a single datastructure. This allows researchers to quickly change approaches when 
data varies in structure from 1 trait to the other.

I the third chapter we showcase our ideas for a generic storage and computation platform. Our demo system xQTL 
workbench is currently being used to run the WormQTL database. xQTL workbench allows users to store and share 
their data, but also run analysis across datasets.

In the last chapter we describe our latest work on using differences in correlation to generate interaction 
networks and mapping these differences back to the genome. This new method has already proven valuable in 
discovering cell type specific eQTL effects seen in gene expression data from taken from whole blood. A 
variation on CTL mapping lead to the discovery that differences in eQTL efect size between samples can be 
attributed to different ratios between the celltypes which make up whole blood.

After Thesis




I hope you'll enjoy reading this thesis as much as I enjoyed working on it for the last couple of years...

Danny Arends (Aug 2013)

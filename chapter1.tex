\chapter{Introduction}
\label{chap:introduction}
A blank piece of paper is God's way of telling us how hard it to be God\\
- Sidney Sheldon (1917 - 2007)\\\\

\emph{This first chapter is to introduce the reader to concepts such as gene, chromosome 
and DNA all the way up to Multiple QTL mapping (MQM) and Genome Wide Association Studies 
(GWAS). This chapter provides the necessary background information for the following 
chapters. Its is however only a short historical overview of systems genetics, and 
introduces only concepts and software used in the analysis of heritable phenotypes that 
are relevant to this thesis. }

\null
\vfill
\newpage

\section{Introduction}
Systems genetics is the interdisciplinairy field which deals with the effects of 
pertubation and natural variation on a biological system. Phenotype variation / 
diversity observed is the consequence of a mixture of variation originating at the 
DNA level (DNA), the environment (Env), and an error component (Err).

$$ Phenotype = Env + DNA + DNA * Env + Err $$

Partitioning of observed variation into this formula is of critical importance 
to understanding the contribution of genetics and environment to the biological 
processes observed. Understanding how a system works and reacts to different 
environments, allows modification of the system to better suit our needs or 
environmental requirements. Knowledge about DNA variation allows to optimize 
breeding of animals and plants, explore the genetic basis of human diseases, 
improved production of chemicals, and so forth.

Variation observed at the DNA level can be measured using genotyping (AFLP, RFLP, 
SNPs, Sequencing) techniques, allowing to create detailed maps of DNA inheritance 
within a population and track the origin of DNA through many generation.

At the phenotype end of the spectrum new high throughput technologies have also 
been developed to mass produce phenotypes measurements using automated phenotyping. 
Automated phenotyping allow thousands of simultanious measurements at all the 
different endophenotype levels such as: metabolome, proteome, transcriptome. These 
new technologies such as tiling arrays, RNAseq, High throughput proteomics and 
metabolomics now enable us to measure all levels from the DNA to the observed 
phenotype at minimal costs.

How the activity of DNA is expressed and modified by signals from the enviroment is the 
focus of systems genetics. The use of computational tools in understanding how genetics 
and environment intertwine is the central in the field of systems genetics to deal with the 
large amounts of phenotype and genotype data.

Variation observed at levels 'further away' from the genome such as: transcriptome, 
proteome, metabolome, all the way up to classical phenotypes form a complex interplay 
between DNA and environment. This interplay has proven to be challenging to unravel.The 
following chapters of this thesis will go into our solutions to these issues, 
infrastructure created, methodologies developed, and their application in current 
research. First some (historical) background to this thesis is be provided.

\section{History (1800-1930)}

The field of genetics is a relatively new field, and is founded when Gregor 
Mendel publishes his work on the inheritance of phenotypes in pea plants. 
His work described in 'Versuche \"uber Pflanzen-Hybriden' in 1865-1866 
\cite{Mendel:1866} gives us Mendels Laws of Inheritance. He observes that 
crossing two plants with different colors of flowers will lead to an offspring 
population with predictable color ratios. He postulated the idea of heredity 
units and stated that each individual carries two of these resulting in a 
single phenotype. Each heritable units is received from one of the parents, 
but which one of the two units the parents pass to a child is random 
(Law 1 - Segregation).

These two units are not equal, a dominant unit (color 1) overruled another 
unit (color 2 - recessive). Mendel studied more phenotypes in his pea plants 
besides color. Now a days it is known that Mendel was fortunate to select 
phenotypes which are caused by a single genes and are inherited independently. 
Mendel also observed this when comparing inheritance of multiple phenotypes. 
He concluded that a unit of inheritance are passed from parent to child in an 
independant fashion in regards to other units of inheritance 
(Law 2 - Independent Assortment).

However Mendels work was not fully recognized until 30 years later, when his work was 
rediscovered by Hugo de Vries, Carl Correns en Erich von Tschermak who (re) define the 
rules for Mendelian Genetics \cite{deVries:1889} providing biologist for the first 
time with a mathematical framework to study heritability of phenotypes caused by a 
single genetic unit.

Twenty years after the rediscovery of Mendels Laws in 1913, careful observations of 
Thomas Hunt Morgan introduces an addition to Mendels theoretical work. He observes 
that some phenotypes in his \emph{Drosophila melanogaster} mutant flies are inherited 
together, in conflict with Mendels Laws (which state independant inheritance of 
genetic units). He calls this phenomenon genetic linkage. This concept is also commonly 
explained as the bead on a string. multiple genetic units linked together. Two beads 
close together are tightly linked are and almost always inherited together. While 
two beads far apart are less linked and can be separated by meiosis. Different 
strands are independantly inherited allowing for Mendelian inheritance by two beads 
on two different strings.

Sturtevant a student of Morgan uses his mentors linkage theory between phenotypes 
as a distance measurement between units of inheritance. He creates the first genetic 
map of \emph{Drosophila melanogaster} 40 years before the discovery of the DNA molecule, 
while molecular mechanisms were still unknown. Sturtevant genetic map based on 
phenotypes allowed genetics based on careful observation of segregating phenotypes in 
large populations of flies, and then determining their distance relative to other 
phenotypes.

\section{DNA and QTLs (1930-1990)}

The dscovery of DNA as the carrier of heritability in 1944 by Avery, MacLeod and McCarty 
\cite{Avery:1944} and the discovery of its structure by Watson and Crick.in 1953 
\cite{Watson:1953} was the long sought after molecular basis for genetics. After the 
discovery of restriction enzymes by Luria and Human \cite{Luria:1952} it was possible 
to cut DNA and investigate the different (lengths of) fragments produced.

In the end of the 1980s everything was there for the next big step in genetics. Three 
papers are published which detail the use of RFLP linkage maps to localize genes 
responsible for variation in quantitative phenotypes. These three papers by David Botstein 
and Eric Lander \cite{Lander:1986, Lander:1987, Lander:1989} form the basis for modern 
day linkage analysis and genome wide association studies. 

Heritable phenotypes could be mapped to genomic locations using a combination of DNA 
restriction enzymes, Mendels inheritance laws and Hunt's linkage theory. Together these 
methodologies and theories provide the experimental and statistical background to 
analyse heritability in any population. Linkage and Association analysis are still the 
foundation of genetics research today. More sophisticated tools and algorithms have been 
developed and are used, but the basic theory remains the same. Some of these more 
advanced methods are discussed in the next section to scetch a background for the work 
presented in this thesis.

\section{Genetical Omics and GWAS (1990-2010)}

Observed phenotype expression is however often more complex then a single causative gene. 
To model these more complex interactions extension of the basic model for QTL mapping is 
necessary. Extending the model can be done by incorporting sources of variation. This allows 
us to associate / partition the observed variance to either environmental factors (E) or a 
genetic loci (G).

Multiple QTL Mapping (MQM) belongs to a family of QTL mapping methods, that include Haley-Knott 
regression \cite{Haley:1992} and composite interval mapping CIM \cite{Zeng:1994}. MQM combines 
the strengths of generalized linear model regression ith those of interval mapping 
\cite{Jansen:1993, Jansen:1994b}. 

R/qtl is an extensible, interactive environment for the mapping of quantitative trait loci (QTL) 
in experimental crosses. It is implemented as an add-on package for the freely available and 
widely used statistical language/software R \cite{R:2009}. Since its introduction, R/qtl 
\cite{Broman:2003} has become a reference implementation with an extensive guide on QTL mapping 
\cite{RQTLGuide:2009}. Its main focus was to provide the mouse community with different QTL 
mapping methodologies, and allow to deal with the the aberrant seggregation X chromosome. 
Furthermore it provides methods to perform quality control of genetic maps. It currently 
supports different types of inbred populations such as BackCross, F2, Recombinant inbread 
lines and 4-way RILs. \cite{Broman:2003}

\begin{table}[h]
  \centering
  {\footnotesize
  \begin{tabular}{ | c | l | l | }
    \hline
    {\bf Molecular level} & {\bf Molecule} & {\bf Technology}\\
    \hline
    \hline
\rowcolor{gray!35}    Genome          & DNA                & RFLP \cite{Lander:1986} \\
\rowcolor{gray!35}    Genome          & DNA                & SNP chips \cite{Hacia:1999} \\
\rowcolor{gray!35}    Genome          & DNA                & DNA sequencing \cite{Mardis:2008} \\
    \hline
    EpiGenome       & DNA methylation    & Bisulfite sequencing \cite{Hayatsu:2007} \\
    EpiGenome       & DNA methylation    & ChIP-on-chip \cite{Collas:2010} \\
    EpiGenome       & DNA methylation    & ChIP-Seq \cite{Park:2009} \\
    \hline
    \hline
\rowcolor{gray!35}    Transcriptome   & RNA          & Microarray \cite{Lashkari:1997}, Tiling array \cite{Lee:2013} \\
\rowcolor{gray!35}    Transcriptome   & RNA          & RNA-Seq \cite{Wang:2009}\\
    \hline
    Proteome        & Proteins     & 2D gel electrophoresis \cite{O'Farrell:1975}\\
    Proteome        & Proteins     & Mass Spectometry \cite{Deshaies:2001}\\
    Proteome        & Proteins     & Antibody protein chip \cite{Fasolo:2009} \\
    \hline
\rowcolor{gray!35}    Metabolome      & Metabolites  & Mass Spectometry \cite{Aebersold:2003} \\
\rowcolor{gray!35}    Metabolome      & Metabolites  & Nuclear magnetic resonance \cite{Espina:2009} \\
    \hline
  \end{tabular}
  }
  \caption{Overview of current technologies}
\end{table}

Genetical Genomics is the concept that views endophenotypes (RNA, Protein and Metabolite abundance) 
as common phenotypes wich can be mapped in bulk to the genome similar to classical 
phenotypes \cite{Jansen:2001a}. Using natural occuring variation and new omics tools allows us 
to track variation from the genotype all the way up to the classical phenotypes. Allowing 
genetics to go from individual QTLs to a system wide approach of analysing QTLs at all known 
molecular levels called Systtems Genetics \cite{Threadgill:2006}. Studies in model organisms 
have shown high heritabilities for endophenotypes such as gene expression making these phenotypes 
ideal targets for QTL mapping \cite{Brem:2002}. 

When mapping gene expression or protein abundance current knowledge of protein and DNA sequence 
allows us to locate their template on the genome. When QTL mapping resolution allows we can even 
distinguish between traits mapping in the proximity of their respective gene (\emph{cis}-eQTL) 
or to other regions in the genome (\emph{trans}-eQTL). This information can be summarized into 
so called cis-trans plots, where the X-axis is the location of the eQTL and the Y-axis the genetic 
location of the trait. Often so called trans-bands are observed, hotspots of many \emph{trans}-
eQTL mapping to a common region in the genome \cite{Breitling:2008a}, which are used to infer 
biological meaning and reconstruction co-expression and/or co-reguatory networks.

In the last decade Genome Wide Association Study (GWAS) have identified thousands of genetic 
variants that are associated with human disease\cite{Hindorff:2009}. For reliable results GWAS 
needs a large cohort of genotyped and phenotyped individuals. Large consortia are working 
together to gather large amounts of human expression data from many different tissues. This 
data is then used in meta analysis leading to eQTL GWA studies with even larger sizes 
(5000+ individuals) \cite{Lude:2011}, leading to more reliable results and enabling discovery 
of new modifiers of human gene expression.

It is known that many factors, such as effects on intermediate molecular phenotypes, influence 
the relationship between genotype and the eventual development of disease. It has also been 
observation that many of the disease-predisposing variant are noncoding, suggests a regulatory 
function for these variants. And it has been shown that many disease-predisposing variants 
(e.g. single nucleotide polymorphisms (SNPs)) affect the expression of nearby genes (i.e. 
\emph{cis}-eQTLs)\cite{Powell:2012, Lude:2011, Zeller:2010}

\section{Thesis contributions (2010-2014)}
Chapter 2 details how Pheno2Geno is developed for high-throughput generation of genetic markers and maps from 
molecular phenotypes. Pheno2Geno selects suitable phenotypes that show clear differential expression 
in the founders. Pheno2Geno uses mixture modelling to select phenotypes showing segregation ratios 
close to the expected mendelian segregation ratios and transform them into genetic markers suitable 
for map construction and/or saturation. Pheno2Geno analyses the candidate genetic markers and excludes 
those showing multiple QTL, epistatically interacting QTL, and QTL by environment interactions to 
provide a set of robust markers for QTL mapping protecting against genetic markers from a non genetic 
origin.

Chapter3 highlights the integration of the Multiple QTL mapping algortihm into R/qtl. In sections 
(3.1 and 3.2) we show the performance of MQM on experimental data from a cross of \emph{A. thaliana} 
Bayreuth x Shahdara.

Chapter4 - Details our work to provide infrastructure for the Life Sciences. We advocate the 
use of Generators to create software and propose a datamodel (XGAP) to store phenotype and 
genotype data. Combining these two approaches we developed: xQTL workbench a scalable web 
platform for the mapping of quantitative trait loci (QTLs) at multiple levels such as gene 
expression (eQTL), protein abundance (pQTL), metabolite abundance (mQTL) and phenotype (phQTL) data. 
Popular QTL mapping methods for model organism and human populations are accessible via the web 
user interface. Large calculations scale easily on to multi-core computers, clusters and Cloud

Chapter5 -  Shows our current work on understanding differences in correlation observed when 
CTL mapping two traits. We show that there is a high overlap between CTL mapping and using an 
G:E interaction model, and use this interaction model in a human GWAS to detect cell type 
specific eQTL.


\chapter{Introduction}
\emph{The introduction of this thesis, background information and the outline of the next chapters.}

\null
\vfill
\newpage

\section{Introduction}
Systems genetics is the inter disciplinairy field which deals with the effects of 
pertubation and natural variation on a biological system. Phenotype variation / 
diversity observed is the consequence of a mixture of variation originating at the 
DNA level (DNA), the environment (Env), and an error component (Err).

$$ Phenotype = Env + DNA + DNA * Env + Err $$

Partitioning of observed variation into this formula is of critical importance 
to understanding the contribution of genetics and environment to the biological 
processes observed. Understanding how a system works and reacts to different 
environments, allows modification of the system to better suit our needs or 
environmental requirements. Knowledge about DNA variation allows to optimize 
breeding of animals and plants, explore the genetic basis of human diseases, 
improved production of chemicals, and so forth.

Variation observed at the DNA level can be measured using genotyping (AFLP, RFLP, 
SNPs, Sequencing) techniques, allowing to create detailed maps of DNA inheritance 
within a population and track the origin of DNA through many generation.

At the phenotype end of the spectrum new high throughput technologies have also 
been developed to mass produce phenotypes measurements using automated phenotyping. 
Automated phenotyping allow thousands of simultanious measurements at all the 
different endophenotype levels such as: metabolome, proteome, transcriptome. These 
new technologies such as tiling arrays, RNAseq, High throughput proteomics and 
metabolomics now enable us to measure all levels from the DNA to the observed 
phenotype at minimal costs.

How the activity of DNA is expressed and modified by signals from the enviroment is the 
focus of systems genetics. The use of computational tools in understanding how genetics 
and environment intertwine is the central in the field of systems genetics to deal with the 
large amounts of phenotype and genotype data.

Variation observed at levels 'further away' from the genome such as: transcriptome, 
proteome, metabolome, all the way up to classical phenotypes form a complex interplay 
between DNA and environment. This interplay has proven to be challenging to unravel.The 
following chapters of this thesis will go into our solutions to these issues, 
infrastructure created, methodologies developed, and their application in current 
research. First some (historical) background to this thesis is be provided.

\subsection{Background 1800-1990}

The field of genetics is a relatively new field, and is founded when Gregor 
Mendel publishes his work on the inheritance of phenotypes in pea plants. 
His work described in 'Versuche \"uber Pflanzen-Hybriden' in 1865-1866 
\cite{Mendel:1866} gives us Mendels Laws of Inheritance. He observes that 
crossing two plants with different colors of flowers will lead to an offspring 
population with predictable color ratios. He postulated the idea of heredity 
units and stated that each individual carries two of these resulting in a 
single phenotype. Each heritable units is received from one of the parents, 
but which one of the two units the parents pass to a child is random 
(Law 1 - Segregation).

These two units are not equal, a dominant unit (color 1) overruled another 
unit (color 2 - recessive). Mendel studied more phenotypes in his pea plants 
besides color. Now a days it is known that Mendel was fortunate to select 
phenotypes which are caused by a single genes and are inherited independently. 
Mendel also observed this when comparing inheritance of multiple phenotypes. 
He concluded that a unit of inheritance are passed from parent to child in an 
independant fashion in regards to other units of inheritance 
(Law 2 - Independent Assortment).

However Mendels work was not fully recognized until 30 years later, when his work was 
rediscovered by Hugo de Vries, Carl Correns en Erich von Tschermak who (re) define the 
rules for Mendelian Genetics \cite{deVries:1889} providing biologist for the first 
time with a mathematical framework to study heritability of phenotypes caused by a 
single genetic unit.

Twenty years after the rediscovery of Mendels Laws in 1913, carefull observations of 
Thomas Hunt Morgan introduces an addition to Mendels theoretical work. He observes 
that some phenotypes in his \emph{Drosophila melanogaster} mutant flies are inherited 
together, in conflict with Mendels Laws (which state independant inheritance of 
genetic units). He calls this phenomenon genetic linkage. This concept is also commonly 
explained as the bead on a string. multiple genetic units linked together. Two beads 
close together are tightly linked are and almost always inherited together. While 
two beads far apart are less linked and can be separated by 'evolution'. Different 
strands are independantly inherited allowing for Mendelian inheritance by two beads 
on two different strings.

Sturtevant a student of Morgan uses his mentors linkage theory between phenotypes 
as a distance measurement between units of inheritance. He creates the first genetic 
map of \emph{Drosophila melanogaster} 40 years before the discovery of the DNA molecule, 
while molecular mechanisms were still unknown. Sturtevant genetic map based on 
phenotypes allowed genetics based on careful observation of segregating phenotypes in 
large populations of flies, and then determining their distance relative to other 
phenotypes.

The dscovery of DNA as the carrier of heritability in 1944 by Avery, MacLeod and McCarty 
\cite{Avery:1944} and the discovery of its structure by Watson and Crick.in 1953 
\cite{Watson:1953} was the long sought after molecular basis for genetics. After the 
discovery of restriction enzymes by Luria and Human \cite{Luria:1952} it was possible 
to cut DNA and investigate the different (lengths of) fragments produced.

In the end of the 1980s everything was there for the next step in genetics. A series of 
three papers is published which detail the use of RFLP linkage maps to localize genes 
responsible for variation in quantitative phenotypes. These three papers by David Botstein 
and Eric Lander \cite{Lander:1986, Lander:1987, Lander:1989} form the basis for modern 
day linkage analysis and genome wide association studies. 

Heritable phenotypes could be mapped to genomic locations using a combination of: DNA 
restriction enzymes, Mendels inheritance laws and Hunt's linkage theory. Together these 
methodologies and theories provide the experimental and statistical background to 
analyse heritability in any population.

Linkage and Association analysis are still the foundation of genetics research today. More 
sophisticated tools and algorithms have been developed and are used, but the basic theory 
remains the same. Some of these more advanced methods are discussed in the next section to 
scetch a background for the work presented in this thesis.

\subsection{Background - Continued}

Multiple QTL mapping - Jansen et al. (1993)\\
R/qtl (2003)\\
Genetical Genomics\\
Human genome project 2000-2003\\
First human Genome Wide Association Study (2005)\\
Encyclopedia Of DNA Elements (Encode) (2003, 2007-Current)

